\documentclass[a4paper,11pt]{article}
\usepackage{fancyhdr}
\usepackage{fancyheadings}
\usepackage[american]{babel}
\usepackage[utf8]{inputenc}
\usepackage[active]{srcltx}
\usepackage{algorithm}
\usepackage[noend]{algorithmic}
\usepackage{amsmath}
\usepackage{amssymb}
\usepackage{amsthm}
\usepackage{bbm}
\usepackage{enumerate}
\usepackage{graphicx}
\usepackage{ifthen}
\usepackage{listings}
\usepackage{struktex}
\usepackage{hyperref}

\usepackage{braket}

\renewcommand{\vector}[2]{{\left(\begin{array}{c} #1 \\ #2 \end{array}\right)}}

%%%%%%%%%%%%%%%%%%%%%%%%%%%%%%%%%%%%%%%%%%%%%%%%%%%%%%
%%%%%%%%%%%%%% EDIT THIS PART %%%%%%%%%%%%%%%%%%%%%%%%
%%%%%%%%%%%%%%%%%%%%%%%%%%%%%%%%%%%%%%%%%%%%%%%%%%%%%%
\newcommand{\Fach}{Basics of Quantum Information and Computing}
\newcommand{\Name}{Michael Hartmann}
\newcommand{\Lehrstuhl}{Theoretische Physik I, Universität Augsburg}
\newcommand{\Uebungsblatt}{4} %  <-- UPDATE ME
\newcommand{\Date}{30.11.2016} %  <-- UPDATE ME
%%%%%%%%%%%%%%%%%%%%%%%%%%%%%%%%%%%%%%%%%%%%%%%%%%%%%%
%%%%%%%%%%%%%%%%%%%%%%%%%%%%%%%%%%%%%%%%%%%%%%%%%%%%%%

\DeclareMathOperator{\Tr}{Tr}

\setlength{\parindent}{0em}
\topmargin -1.0cm
\oddsidemargin 0cm
\evensidemargin 0cm
\setlength{\textheight}{9.2in}
\setlength{\textwidth}{6.0in}

%%%%%%%%%%%%%%%
%% Problem-COMMAND
\newcommand{\Problem}[1]{
  {
  \vspace*{0.5cm}
  \textsf{\textbf{Problem #1}}
  \vspace*{0.2cm}
  
  }
}
%%%%%%%%%%%%%%
\hypersetup{
    pdftitle={\Fach{}: Exercise \Uebungsblatt{}},
    pdfauthor={\Name},
    pdfborder={0 0 0}
}

\lstset{ %
language=java,
basicstyle=\footnotesize\tt,
showtabs=false,
tabsize=2,
captionpos=b,
breaklines=true,
extendedchars=true,
showstringspaces=false,
flexiblecolumns=true,
}

\title{Exercise \Uebungsblatt{}}
\author{\Name{}}

\begin{document}
\thispagestyle{fancy}
\lhead{\sf \Fach{} \\ \tiny{\Name, \Lehrstuhl}}
\rhead{\sf \Date{}}
\vspace*{0.2cm}
\begin{center}
\LARGE \sf \textbf{Exercise \Uebungsblatt{} -- Kronecker Product, Walsh-Hadamard Transform, Quantum Teleportation}
\end{center}
\vspace*{0.2cm}

%%%%%%%%%%%%%%%%%%%%%%%%%%%%%%%%%%%%%%%%%%%%%%%%%%%%%%
%% Insert your solutions here %%%%%%%%%%%%%%%%%%%%%%%%
%%%%%%%%%%%%%%%%%%%%%%%%%%%%%%%%%%%%%%%%%%%%%%%%%%%%%%
\Problem{1}
Let $A \equiv (a_{ij})_{ij}$ be an $m \times n$ matrix and $B$ an $r \times s$
matrix. The Kronecker product of A and B is defined as the $(m \cdot r) \times (n \cdot s)$
matrix
\begin{equation}
A \otimes B = \left(
\begin{array}{cccc}
a_{11} B & a_{12} B & \hdots & a_{1n} B \\
a_{21} B & a_{22} B & \hdots & a_{2n} B \\
\vdots   & \vdots   & \ddots & \vdots   \\
a_{m1} B & a_{m2} B & \hdots & a_{mn} B \\
\end{array}
\right).
\end{equation}

\begin{enumerate}[a)]
    \item The states
        \begin{equation}
        \ket{\phi_1} \equiv \vector{1}{0}, \quad \ket{\phi_2} \equiv \vector{0}{1}
        \end{equation}
    form a basis of $\mathbb{C}^2$. Calculate
    \begin{equation}
    \ket{\phi_1} \otimes \ket{\phi_1}, \quad \ket{\phi_1} \otimes \ket{\phi_2}, \quad \ket{\phi_2} \otimes \ket{\phi_1}, \quad \ket{\phi_2} \otimes \ket{\phi_2}.
    \end{equation}
    Form these vectors also a basis of $\mathbb{C}^4$?

    \item Consider the Pauli matrices
    \begin{equation}
    \sigma_x = \left(\begin{array}{cc} 0 & 1 \\ 1 & 0 \end{array}\right), \quad \sigma_z = \left(\begin{array}{cc} 1 & 0 \\ 0 & -1 \end{array}\right).
    \end{equation}
    Find $\sigma_x \otimes \sigma_z$ and $\sigma_z \otimes \sigma_x$.
\end{enumerate}

\Problem{2}
The single-bit {\it Walsh-Hadamard transform} is the unitary map $W_1$ given by
\begin{equation}
W_1\ket{0} = \frac{1}{\sqrt 2} \left(\ket{0}+\ket{1}\right), \qquad W_1\ket{1} = \frac{1}{\sqrt 2} \left(\ket{0}-\ket{1}\right)
\end{equation}
The $n$-bit Walsh-Hadamard transform $W_n$ is defined as
\begin{equation}
W_n \equiv W_1 \otimes W_1 \otimes \dots \otimes W_1 \qquad \text{($n$-times)}.
\end{equation}
\begin{enumerate}[a)]
\item Find $W_2$ using the canonical basis
\begin{equation}
\ket{0} = \vector{1}{0}, \quad \ket{1} = \vector{0}{1}.
\end{equation}
\item Find the inverse of $W_2$.
\item Find $W_2 \left(\ket{00}\right)$.
\item Find $W_n \left(\ket{00\dots0}\right)$.
\end{enumerate}


\Problem{3}

Suppose Alice has a pure state $\ket{\psi}_S$ of a system $S$ and wants to send
it to Bob. Using a shared entangled state, she can ``teleport'' the state
$\ket{\psi}_S$ without having to physically move the state over.

We have three systems $\mathcal{H}_S \otimes \mathcal{H}_A \otimes
\mathcal{H}_B$. The initial state is given by
\begin{equation}
\ket{\psi}_\text{in} = \ket{\psi}_S \otimes \frac{1}{\sqrt 2} \left(\ket{0_A 0_B} + \ket{1_A 1_B}\right),
\end{equation}
i.e. Alice and Bob share a fully entangled Bell state, and the total state is
separable with respect to $S$. We may write $\ket{\psi}_S=\alpha\ket{0}+\beta\ket{1}$.

\begin{enumerate}[a)]
\item
In a first step, Alice will measure systems $S$ and $A$ jointly in the Bell basis,
\begin{equation}
\left\{\begin{array}{lr}
\frac{1}{\sqrt 2} \left( \ket{0_S 0_A} + \ket{1_S 1_A} \right) & \frac{1}{\sqrt 2}  \left( \ket{0_S 0_A} - \ket{1_S 1_A} \right)  \\
\frac{1}{\sqrt 2} \left( \ket{0_S 1_A} + \ket{1_S 0_A} \right) & \frac{1}{\sqrt 2}  \left( \ket{0_S 1_A} - \ket{1_S 0_A} \right) 
\end{array}\right\}.
\end{equation}
What is the reduced state of Bob's system for each of the possible
outcomes? What are the probabilities for the possible outcomes?

\item Alice then (classically) communicates the result of her measurement to
Bob.  What operations can Bob apply to his system to recover $\ket\psi_S$?

\item Suppose that Alice does not manage to tell Bob the outcome of her
measurement. Show that in this case he does not have any information about his
reduced state and therefore does not know which operation to apply in order to
obtain $\ket\psi_S$.

\end{enumerate}

\end{document}
