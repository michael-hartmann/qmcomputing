\documentclass[a4paper,10pt]{article}
\usepackage{fancyhdr}
\usepackage{fancyheadings}
\usepackage[american]{babel}
\usepackage[utf8]{inputenc}
\usepackage[active]{srcltx}
\usepackage{algorithm}
\usepackage[noend]{algorithmic}
\usepackage{amsmath}
\usepackage{amssymb}
\usepackage{amsthm}
\usepackage{bbm}
\usepackage{enumerate}
\usepackage{graphicx}
\usepackage{ifthen}
\usepackage{listings}
\usepackage{struktex}
\usepackage{hyperref}

\usepackage{braket}

\renewcommand{\vector}[2]{{\left(\begin{array}{c} #1 \\ #2 \end{array}\right)}}

%%%%%%%%%%%%%%%%%%%%%%%%%%%%%%%%%%%%%%%%%%%%%%%%%%%%%%
%%%%%%%%%%%%%% EDIT THIS PART %%%%%%%%%%%%%%%%%%%%%%%%
%%%%%%%%%%%%%%%%%%%%%%%%%%%%%%%%%%%%%%%%%%%%%%%%%%%%%%
\newcommand{\Fach}{Basics of Quantum Information and Computing}
\newcommand{\Name}{Michael Hartmann}
\newcommand{\Lehrstuhl}{Theoretische Physik I, Universität Augsburg}
\newcommand{\Uebungsblatt}{9} %  <-- UPDATE ME
\newcommand{\Date}{27.01.2017} %  <-- UPDATE ME
%%%%%%%%%%%%%%%%%%%%%%%%%%%%%%%%%%%%%%%%%%%%%%%%%%%%%%
%%%%%%%%%%%%%%%%%%%%%%%%%%%%%%%%%%%%%%%%%%%%%%%%%%%%%%

\DeclareMathOperator{\Tr}{Tr}

\setlength{\parindent}{0em}
\setlength{\parskip}{1em}
\topmargin -1.0cm
\oddsidemargin 0cm
\evensidemargin 0cm
\setlength{\textheight}{9.2in}
\setlength{\textwidth}{6.0in}

%%%%%%%%%%%%%%%
%% Problem-COMMAND
\newcommand{\Problem}[1]{
  {
  \vspace*{0.5cm}
  \textsf{\textbf{Problem #1}}
  \vspace*{0.2cm}
  
  }
}
%%%%%%%%%%%%%%
\hypersetup{
    pdftitle={\Fach{}: Exercise \Uebungsblatt{}},
    pdfauthor={\Name},
    pdfborder={0 0 0}
}

\lstset{ %
language=java,
basicstyle=\footnotesize\tt,
showtabs=false,
tabsize=2,
captionpos=b,
breaklines=true,
extendedchars=true,
showstringspaces=false,
flexiblecolumns=true,
}

\title{Exercise \Uebungsblatt{}}
\author{\Name{}}

\begin{document}
\thispagestyle{fancy}
\lhead{\sf \Fach{} \\ \tiny{\Name, \Lehrstuhl}}
\rhead{\sf \Date{}}
\vspace*{0.2cm}
\begin{center}
\LARGE \sf \textbf{Exercise \Uebungsblatt{} -- Grover's algorithm}
\end{center}
\vspace*{0.2cm}

%%%%%%%%%%%%%%%%%%%%%%%%%%%%%%%%%%%%%%%%%%%%%%%%%%%%%%
%% Insert your solutions here %%%%%%%%%%%%%%%%%%%%%%%%
%%%%%%%%%%%%%%%%%%%%%%%%%%%%%%%%%%%%%%%%%%%%%%%%%%%%%%

Suppose we whish to search through a search space of $N$ elements. Rather than
to search the elements directly, we concentrate on the {\it index} to those
elements, which is just a number in the range $0$ to $N-1$. We assume that $N$ is a
power of $2$, $N=2^n$, and that the search problem has exactly one solution.

The Grover's algorithm is summarized as follows:
\begin{enumerate}
\item Prepare the initial state $\ket{0}^{\otimes n}$.
\item Apply the Hadamard transform to all qubits:
\begin{equation}
\ket{\psi} = H^{\otimes n} \ket{0}^{\otimes n}
\end{equation}
\item Apply the Grover iteration $\mathcal{G}$ approximately $R\approx\frac{\pi}{4}\sqrt{N}$ times, where
\begin{equation}
\mathcal{G} = \left(2\ket{\psi}\bra{\psi} - \mathbbm{1} \right) \mathcal{O},
\end{equation}
and $\mathcal{O}$ is a quantum oracle as described in problem 1.
\item Measure the register $x_0$.
\end{enumerate}

\Problem{1 -- Quantum oracle}

Suppose we are supplied with a quantum oracle. More precisely, the oracle is a
unitary operator $\mathcal{O}$, defined by its action on the computational
basis:
\begin{equation}
\ket{x} \ket{q} \overset{\mathcal{O}}{\rightarrow} \ket{x} \ket{q \oplus f(x)},
\end{equation}
where $\ket{x}$ is the index register, $\ket{q}$ is the oracle qubit, $\oplus$ denotes addition modulo $2$, and $f(x)=1$ if $x$ is a solution
to the search problem, and $f(x)=0$ if $x$ is not a solution to the search problem.

\begin{enumerate}[a)]
\item Calculate the action of the oracle on an element of the search space if
the oracle qubit was originally in the state $(\ket{0}-\ket{1})/\sqrt{2}$.
\item Show that the state of the oracle qubit is not changed.
\item Show that the oracle marks the solution to the search problem by shifting
the phase of the solution.
\end{enumerate}

\Problem{2 -- Grover's algorithm}

\begin{enumerate}[a)]
\item Consider the normalized states
\begin{equation}
\ket{\alpha} \equiv \frac{1}{\sqrt{N-1}} \sum_{x,x\ne \beta} \ket{x}, \qquad \ket{\beta},
\end{equation}
where $\beta$ is the solution to the search problem, and
\begin{equation}
\ket{\psi} = \sqrt{\frac{N-1}{N}} \ket{\alpha} + \frac{1}{\sqrt{N}} \ket{\beta}.
\end{equation}
Show that in the
$\ket{\alpha}$, $\ket{\beta}$ basis, we may write the Grover iteration as
\begin{equation}
\mathcal{G}= \left(\begin{array}{cc}
\cos\theta & -\sin\theta \\
\sin\theta &  \cos\theta
\end{array}\right).
\end{equation}
\item Calculate $\mathcal{G}^k\ket{\psi}$.
\item Show that the Grover iteration has to be repeated approximately $R\approx\frac{\pi}{4}\sqrt{N}$ times for $N \gg 1$.
\item What is the probability that the algorithm succeeds for $N=8$, $N=128$, and
$N=1024$?
\item What is the minimum, maximum and average number of comparisons needed by
a classical linear search? What is the complexity of this algorithm?
\item What is the complexity to search a sorted list?
\end{enumerate}

\end{document}
