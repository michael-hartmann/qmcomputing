\documentclass[a4paper,11pt]{article}
\usepackage{fancyhdr}
\usepackage{fancyheadings}
\usepackage[american]{babel}
\usepackage[utf8]{inputenc}
\usepackage[active]{srcltx}
\usepackage{algorithm}
\usepackage[noend]{algorithmic}
\usepackage{amsmath}
\usepackage{amssymb}
\usepackage{amsthm}
\usepackage{bbm}
\usepackage{enumerate}
\usepackage{graphicx}
\usepackage{ifthen}
\usepackage{listings}
\usepackage{struktex}
\usepackage{hyperref}

\usepackage{braket}

\renewcommand{\vector}[2]{{\left(\begin{array}{c} #1 \\ #2 \end{array}\right)}}

%%%%%%%%%%%%%%%%%%%%%%%%%%%%%%%%%%%%%%%%%%%%%%%%%%%%%%
%%%%%%%%%%%%%% EDIT THIS PART %%%%%%%%%%%%%%%%%%%%%%%%
%%%%%%%%%%%%%%%%%%%%%%%%%%%%%%%%%%%%%%%%%%%%%%%%%%%%%%
\newcommand{\Fach}{Basics of Quantum Information and Computing}
\newcommand{\Name}{Michael Hartmann}
\newcommand{\Lehrstuhl}{Theoretische Physik I, Universität Augsburg}
\newcommand{\Uebungsblatt}{5} %  <-- UPDATE ME
\newcommand{\Date}{5.12.2016} %  <-- UPDATE ME
%%%%%%%%%%%%%%%%%%%%%%%%%%%%%%%%%%%%%%%%%%%%%%%%%%%%%%
%%%%%%%%%%%%%%%%%%%%%%%%%%%%%%%%%%%%%%%%%%%%%%%%%%%%%%

\DeclareMathOperator{\Tr}{Tr}

\setlength{\parindent}{0em}
\topmargin -1.0cm
\oddsidemargin 0cm
\evensidemargin 0cm
\setlength{\textheight}{9.2in}
\setlength{\textwidth}{6.0in}

%%%%%%%%%%%%%%%
%% Problem-COMMAND
\newcommand{\Problem}[1]{
  {
  \vspace*{0.5cm}
  \textsf{\textbf{Problem #1}}
  \vspace*{0.2cm}
  
  }
}
%%%%%%%%%%%%%%
\hypersetup{
    pdftitle={\Fach{}: Exercise \Uebungsblatt{}},
    pdfauthor={\Name},
    pdfborder={0 0 0}
}

\lstset{ %
language=java,
basicstyle=\footnotesize\tt,
showtabs=false,
tabsize=2,
captionpos=b,
breaklines=true,
extendedchars=true,
showstringspaces=false,
flexiblecolumns=true,
}

\title{Exercise \Uebungsblatt{}}
\author{\Name{}}

\begin{document}
\thispagestyle{fancy}
\lhead{\sf \Fach{} \\ \tiny{\Name, \Lehrstuhl}}
\rhead{\sf \Date{}}
\vspace*{0.2cm}
\begin{center}
\LARGE \sf \textbf{Exercise \Uebungsblatt{} -- Density Operator}
\end{center}
\vspace*{0.2cm}

The \textit{density operator} $\varrho$ is defined as
\begin{equation}
\varrho \equiv \sum_{j} p_j \ket{\Psi_j}\bra{\Psi_j}, \qquad \sum_j p_j = 1 \,.
\end{equation}
The system may be found in the state $\ket{\Psi_j}$ with probability $p_j$.
If the state of the system is known exactly, i.e., $\varrho=\ket{\Psi}\bra{\Psi}$, the
state is called \textit{pure}, otherwise \textit{mixed}.

Suppose we have a system $\varrho^{AB}$ which consists of two subsystems $A$
and $B$. The \textit{reduced density operator} is defined by
\begin{equation}
\varrho^{A} \equiv \Tr_B\left(\varrho^{AB}\right),
\end{equation}
where
\begin{equation}
\Tr_B\left(\ket{a_1}\bra{a_2} \otimes \ket{b_1}\bra{b_2}\right) \equiv \ket{a_1}\bra{a_2} \Tr\left(\ket{b_1}\bra{b_2}\right),
\end{equation}
and $\ket{a_1},\ket{a_2}$ are states of $A$, and $\ket{b_1},\ket{b_2}$ are states of $B$.


\Problem{1}
Proof that
\begin{enumerate}[a)]
\item $\Tr\varrho = 1$,
\item $\varrho$ is Hermitian,
\item $\varrho$ is semi-positive,
\item $\Tr\varrho^2 \le 1$ and $\Tr\varrho^2=1$ if and only if $\varrho$ is a pure state.
\end{enumerate}


\Problem{2}
\begin{enumerate}[a)]
\item Show that the density operator $\varrho$ fulfills the \textit{von Neumann} equation
\begin{equation}
\dot \varrho = -\frac{i}{\hbar} \left[H,\varrho\right] \,.
\end{equation}

\item Show that an arbitrary density matrix for a mixed state qubit may be written as
\begin{equation}
\varrho = \frac{1}{2} \left(\mathbbm{1} + \vec r \cdot \vec \varrho \right) \,.
\end{equation}
For what $\vec r$ is $\varrho$ pure?
\end{enumerate}

\Problem{3}
Consider the state
\begin{equation}
\varrho^{12} = \left(\frac{\ket{00}+\ket{11}}{\sqrt{2}}\right) \left(\frac{\bra{00}+\bra{11}}{\sqrt{2}}\right)
\end{equation}
\begin{enumerate}[a)]
\item Show that $\varrho^{12}$ is a pure state.
\item Calculate $\varrho^1$ and show that $\varrho^1$ is a mixed state.
\end{enumerate}

\Problem{4}
Suppose a composite system of $A$ and $B$ is in the state $\ket{a}\ket{b}$,
where $\ket{a}$ is a pure state of system $A$, and $\ket{b}$ is a pure state of
system $B$. Show that the reduced density operator of system $A$ alone is a
pure state.

\Problem{5}
In exercise 5 the initial state was given by
\begin{equation}
\ket{\Psi}_\text{in} = \frac{1}{\sqrt{2}} \left( \alpha\ket{0_S 0_A 0_B} + \alpha\ket{0_S 1_A 1_B} + \beta\ket{1_S 0_A 0_B} +\beta\ket{1_S 1_A 1_B} \right) \,.
\end{equation}
Calculate the reduced density operator $\varrho^B$ and show that it is a mixed state.

\end{document}
