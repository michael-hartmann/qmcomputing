\documentclass[a4paper,10pt]{article}
\usepackage{fancyhdr}
\usepackage{fancyheadings}
\usepackage[american]{babel}
\usepackage[utf8]{inputenc}
\usepackage[active]{srcltx}
\usepackage{algorithm}
\usepackage[noend]{algorithmic}
\usepackage{amsmath}
\usepackage{amssymb}
\usepackage{amsthm}
\usepackage{bbm}
\usepackage{enumerate}
\usepackage{graphicx}
\usepackage{ifthen}
\usepackage{listings}
\usepackage{struktex}
\usepackage{hyperref}

\usepackage{braket}

\renewcommand{\vector}[2]{{\left(\begin{array}{c} #1 \\ #2 \end{array}\right)}}

%%%%%%%%%%%%%%%%%%%%%%%%%%%%%%%%%%%%%%%%%%%%%%%%%%%%%%
%%%%%%%%%%%%%% EDIT THIS PART %%%%%%%%%%%%%%%%%%%%%%%%
%%%%%%%%%%%%%%%%%%%%%%%%%%%%%%%%%%%%%%%%%%%%%%%%%%%%%%
\newcommand{\Fach}{Basics of Quantum Information and Computing}
\newcommand{\Name}{Michael Hartmann}
\newcommand{\Lehrstuhl}{Theoretische Physik I, Universität Augsburg}
\newcommand{\Uebungsblatt}{8} %  <-- UPDATE ME
\newcommand{\Date}{20.01.2016} %  <-- UPDATE ME
%%%%%%%%%%%%%%%%%%%%%%%%%%%%%%%%%%%%%%%%%%%%%%%%%%%%%%
%%%%%%%%%%%%%%%%%%%%%%%%%%%%%%%%%%%%%%%%%%%%%%%%%%%%%%

\DeclareMathOperator{\Tr}{Tr}

\setlength{\parindent}{0em}
\setlength{\parskip}{1em}
\topmargin -1.0cm
\oddsidemargin 0cm
\evensidemargin 0cm
\setlength{\textheight}{9.2in}
\setlength{\textwidth}{6.0in}

%%%%%%%%%%%%%%%
%% Problem-COMMAND
\newcommand{\Problem}[1]{
  {
  \vspace*{0.5cm}
  \textsf{\textbf{Problem #1}}
  \vspace*{0.2cm}
  
  }
}
%%%%%%%%%%%%%%
\hypersetup{
    pdftitle={\Fach{}: Exercise \Uebungsblatt{}},
    pdfauthor={\Name},
    pdfborder={0 0 0}
}

\lstset{ %
language=java,
basicstyle=\footnotesize\tt,
showtabs=false,
tabsize=2,
captionpos=b,
breaklines=true,
extendedchars=true,
showstringspaces=false,
flexiblecolumns=true,
}

\title{Exercise \Uebungsblatt{}}
\author{\Name{}}

\begin{document}
\thispagestyle{fancy}
\lhead{\sf \Fach{} \\ \tiny{\Name, \Lehrstuhl}}
\rhead{\sf \Date{}}
\vspace*{0.2cm}
\begin{center}
\LARGE \sf \textbf{Exercise \Uebungsblatt{}}
\end{center}
\vspace*{0.2cm}

%%%%%%%%%%%%%%%%%%%%%%%%%%%%%%%%%%%%%%%%%%%%%%%%%%%%%%
%% Insert your solutions here %%%%%%%%%%%%%%%%%%%%%%%%
%%%%%%%%%%%%%%%%%%%%%%%%%%%%%%%%%%%%%%%%%%%%%%%%%%%%%%

\Problem{1 -- $Z$-$Y$ decomposition of a single qubit}

The Pauli matrices are given by
\begin{equation}
\sigma_x = \left(
\begin{array}{cc}
0 & 1 \\
1 & 0
\end{array}
\right), \quad
\sigma_y = \left(
\begin{array}{cc}
0 & -i \\
i & 0
\end{array}
\right), \quad
\sigma_z = \left(
\begin{array}{cc}
1 & 0 \\
0 & -1
\end{array}
\right).
\end{equation}

We will show that any unitary gate on a single qubit can be implemented using
only $Z$ and $Y$ rotations.

\begin{enumerate}[a)]

\item Show that $\sigma_x^2 = \sigma_y^2 = \sigma_z^2 = \mathbbm{1}$.

\item Show that, if $A$ is a matrix such that $A^2=\mathbbm{1}$, then, for any real number $x$,
\begin{equation}
e^{ixA} = \cos{x} \mathbbm{1} + i \sin{x} A.
\end{equation}
\item Use the previous step to show that
\begin{align}
R_y(\theta) &\equiv e^{-i\frac{\theta}{2}\sigma_y} = \left(\begin{array}{cc} \cos{\frac{\theta}{2}} & -\sin{\frac{\theta}{2}} \\ \sin{\frac{\theta}{2}} & \cos{\frac{\theta}{2}} \end{array}\right) \\
R_z(\theta) &\equiv e^{-i\frac{\theta}{2}\sigma_z} = \left(\begin{array}{cc} e^{-i\frac{\theta}{2}} & 0 \\ 0 & e^{i\frac{\theta}{2}} \end{array}\right) 
\end{align}

\item Show that a unitary $2 \times 2$ matrix can be written as
\begin{equation}
U = \left(
\begin{array}{cc}
e^{i(\alpha-\beta-\delta)}\cos\gamma & -e^{i(\alpha-\beta+\delta)}\sin\gamma \\
e^{i(\alpha+\beta-\delta)}\sin\gamma &  e^{i(\alpha+\beta+\delta)}\cos\gamma
\end{array}
\right)
\end{equation}
where $\alpha,\beta,\gamma,\delta$ are real numbers.

\item Use the results above to show that $U$ can be implemented as
\begin{equation}
U = e^{i\alpha} R_z(2\beta) R_y(2\gamma) R_z(2\delta).
\end{equation}
\end{enumerate}


\end{document}
