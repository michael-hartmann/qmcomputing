\documentclass[a4paper,10pt]{article}
\usepackage{fancyhdr}
\usepackage{fancyheadings}
\usepackage[american]{babel}
\usepackage[utf8]{inputenc}
\usepackage[active]{srcltx}
\usepackage{algorithm}
\usepackage[noend]{algorithmic}
\usepackage{amsmath}
\usepackage{amssymb}
\usepackage{amsthm}
\usepackage{bbm}
\usepackage{enumerate}
\usepackage{graphicx}
\usepackage{ifthen}
\usepackage{listings}
\usepackage{struktex}
\usepackage{hyperref}

\usepackage{braket}

\renewcommand{\vector}[2]{{\left(\begin{array}{c} #1 \\ #2 \end{array}\right)}}

%%%%%%%%%%%%%%%%%%%%%%%%%%%%%%%%%%%%%%%%%%%%%%%%%%%%%%
%%%%%%%%%%%%%% EDIT THIS PART %%%%%%%%%%%%%%%%%%%%%%%%
%%%%%%%%%%%%%%%%%%%%%%%%%%%%%%%%%%%%%%%%%%%%%%%%%%%%%%
\newcommand{\Fach}{Basics of Quantum Information and Computing}
\newcommand{\Name}{Michael Hartmann}
\newcommand{\Lehrstuhl}{Theoretische Physik I, Universität Augsburg}
\newcommand{\Uebungsblatt}{10} %  <-- UPDATE ME
\newcommand{\Date}{10.02.2016} %  <-- UPDATE ME
%%%%%%%%%%%%%%%%%%%%%%%%%%%%%%%%%%%%%%%%%%%%%%%%%%%%%%
%%%%%%%%%%%%%%%%%%%%%%%%%%%%%%%%%%%%%%%%%%%%%%%%%%%%%%

\DeclareMathOperator{\Tr}{Tr}
\DeclareMathOperator{\vect}{vec}

\setlength{\parindent}{0em}
\setlength{\parskip}{1em}
\topmargin -1.0cm
\oddsidemargin 0cm
\evensidemargin 0cm
\setlength{\textheight}{9.2in}
\setlength{\textwidth}{6.0in}

%%%%%%%%%%%%%%%
%% Problem-COMMAND
\newcommand{\Problem}[1]{
  {
  \vspace*{0.5cm}
  \textsf{\textbf{Problem #1}}
  \vspace*{0.2cm}
  
  }
}
%%%%%%%%%%%%%%
\hypersetup{
    pdftitle={\Fach{}: Exercise \Uebungsblatt{}},
    pdfauthor={\Name},
    pdfborder={0 0 0}
}

\lstset{ %
language=java,
basicstyle=\footnotesize\tt,
showtabs=false,
tabsize=2,
captionpos=b,
breaklines=true,
extendedchars=true,
showstringspaces=false,
flexiblecolumns=true,
}

\title{Exercise \Uebungsblatt{}}
\author{\Name{}}

\begin{document}
\thispagestyle{fancy}
\lhead{\sf \Fach{} \\ \tiny{\Name, \Lehrstuhl}}
\rhead{\sf \Date{}}
\vspace*{0.2cm}
\begin{center}
\LARGE \sf \textbf{Exercise \Uebungsblatt{}}
\end{center}
\vspace*{0.2cm}

%%%%%%%%%%%%%%%%%%%%%%%%%%%%%%%%%%%%%%%%%%%%%%%%%%%%%%
%% Insert your solutions here %%%%%%%%%%%%%%%%%%%%%%%%
%%%%%%%%%%%%%%%%%%%%%%%%%%%%%%%%%%%%%%%%%%%%%%%%%%%%%%


\Problem{1 -- Von-Neumann equation}

Show that the derivation of the density operator $\varrho$ in the Schrödinger picture is given by the
Von-Neumann equation
\begin{equation}
\frac{\partial\varrho}{\partial t} = -\frac{i}{\hbar} \big[H,\varrho\big] \,.
\end{equation}


\Problem{2 -- Lindblad equation}
A master equation in Lindblad form
\begin{equation}
\frac{\partial\varrho}{\partial t} = -\frac{i}{\hbar} \big[H,\varrho\big] + \sum_{j=1}^{N^2-1} \gamma_j \left( A_j\varrho A_j^\dagger - \frac{1}{2}\varrho A_j^\dagger A_j - \frac{1}{2}A_j^\dagger A_j \varrho \right), \qquad \gamma_j \ge 0
\end{equation}
is the most general type of a Markovian and time-homogeneous master equation
describing non-unitary evolution of the density matrix $\varrho$ that is
trace-preserving and completely positive for any initial condition. The
operators $A_j$ are called jump operators and are operators of the Hilbert
space.

We consider the Hamiltonian
\begin{equation}
H = \frac{\hbar\omega}{2} \sigma_x
\end{equation}
with jump operators
\begin{equation}
\sigma_+ = \frac{1}{2} \left( \sigma_x + i\sigma_y \right), \qquad
\sigma_- = \frac{1}{2} \left( \sigma_x - i\sigma_y \right)
\end{equation}
and associated rates $\gamma_+$ and $\gamma_-$.

\begin{enumerate}[i)]
\item Calculate $\sigma_+\sigma_-$ and $\sigma_-\sigma_+$, the commutator $[\sigma_+,\sigma_-]$, and the anti-commutator $\{\sigma_+,\sigma_-\}$.
\item The Kronecker product can be used to get a convenient representation for some matrix equations,
\begin{equation}
A\varrho B = C \quad \Leftrightarrow \quad (B^T \otimes A) \vect(\varrho) = \vect(A\varrho B) = \vect(\varrho) \equiv \vec \varrho .
\end{equation}
Here, $\vect(\varrho)$ denotes the vectorization of the matrix $\varrho$ formed by stacking the columns of $\varrho$ into a single column vector,
\begin{equation}
\vect\left(\begin{array}{cc} a & b \\ b^* & c \end{array}\right) = \left(\begin{array}{l} a \\ b^* \\ b \\ c \end{array}\right).
\end{equation}
Calculate the $4\times4$ matrix $\mathcal{L}$ such that
\begin{equation}
\frac{\mathrm{d}}{\mathrm{d}t} \vec\varrho = \mathcal{L}\vec \varrho.
\end{equation}
\item Show that the Lindblad equation for this system is trace preserving.
\item Calculate the steady state of the system for the rates given by $\gamma_+
= \gamma(\nu-1)$ and $\gamma_- = \gamma\nu$.  What is the interpretation of
$\gamma$ and $\nu$?
\end{enumerate}


\Problem{3 -- Phase-damping channel}

Define a quantum operation $\mathcal{E}$ on a qubit state as
\begin{equation}
\mathcal{E}(\varrho) = E_0 \varrho E_0^\dagger + E_1 \varrho E_1^\dagger
\end{equation}
with
\begin{equation}
E_0 = \left(\begin{array}{cc}
1 & 0 \\
0 & \sqrt{1-p}
\end{array}\right), \qquad
E_1 = \left(\begin{array}{cc}
0 & 0 \\
0 & \sqrt{p}
\end{array}\right),
\end{equation}
and $0 < p < 1$.

\begin{enumerate}[i)]
\item Prove that $\mathcal{E}$ maps a density matrix to a density matrix.
\item Show that for any
$\varrho = \left(\begin{array}{cc} a   & b \\ b^* & c \end{array}\right)$
\begin{equation}
\lim_{N\to\infty} \mathcal{E}^N (\varrho) = \left(\begin{array}{cc}
a & 0 \\
0 & c
\end{array}\right).
\end{equation}
\end{enumerate}


\vfill
\hrulefill \\
Pauli matrices:
\begin{equation}
\sigma_0 = \mathbbm{1}, \qquad
\sigma_x = \left(\begin{array}{cc} 0 & 1 \\ 1 & 0 \end{array}\right), \qquad
\sigma_y = \left(\begin{array}{cc} 0 & -i \\ i & 0 \end{array}\right), \qquad
\sigma_z = \left(\begin{array}{cc} 1 & 0 \\ 0 & -1 \end{array}\right)
\end{equation}

\end{document}
