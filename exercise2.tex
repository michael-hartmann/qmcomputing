\documentclass[a4paper,12pt]{article}
\usepackage{fancyhdr}
\usepackage{fancyheadings}
\usepackage[english]{babel}
\usepackage[utf8]{inputenc}
\usepackage[active]{srcltx}
\usepackage{amsmath}
\usepackage{amssymb}
\usepackage{amsthm}
\usepackage{bbm}
\usepackage{enumerate}
\usepackage{hyperref}

\usepackage{braket}

\renewcommand{\vector}[2]{{\left(\begin{array}{c} #1 \\ #2 \end{array}\right)}}

%%%%%%%%%%%%%%%%%%%%%%%%%%%%%%%%%%%%%%%%%%%%%%%%%%%%%%
%%%%%%%%%%%%%% EDIT THIS PART %%%%%%%%%%%%%%%%%%%%%%%%
%%%%%%%%%%%%%%%%%%%%%%%%%%%%%%%%%%%%%%%%%%%%%%%%%%%%%%
\newcommand{\Fach}{Basics of Quantum Information and Computing}
\newcommand{\Name}{Michael Hartmann}
\newcommand{\Lehrstuhl}{Theoretische Physik I}
\newcommand{\Uebungsblatt}{2}  %  <-- UPDATE ME
\newcommand{\Date}{23.11.2016} %  <-- UPDATE ME
%%%%%%%%%%%%%%%%%%%%%%%%%%%%%%%%%%%%%%%%%%%%%%%%%%%%%%
%%%%%%%%%%%%%%%%%%%%%%%%%%%%%%%%%%%%%%%%%%%%%%%%%%%%%%

\DeclareMathOperator{\Tr}{tr}

\setlength{\parindent}{0em}
\topmargin -1.0cm
\oddsidemargin 0cm
\evensidemargin 0cm
\setlength{\textheight}{9.2in}
\setlength{\textwidth}{6.0in}

%%%%%%%%%%%%%%%
%% Problem-COMMAND
\newcommand{\Problem}[1]{
  {
  \vspace*{0.5cm}
  \textsf{\textbf{Problem #1}}
  \vspace*{0.2cm}
  
  }
}
%%%%%%%%%%%%%%
\hypersetup{
    pdftitle={\Fach{}: Exercise \Uebungsblatt{}},
    pdfauthor={\Name},
    pdfborder={0 0 0}
}

\title{Exercise \Uebungsblatt{}}
\author{\Name{}}

\begin{document}
\thispagestyle{fancy}
\lhead{\sf \Fach{} \\ \tiny{\Name, \Lehrstuhl}}
\rhead{\sf \Date{}}
\vspace*{0.2cm}
\begin{center}
\LARGE \sf \textbf{Exercise \Uebungsblatt{} -- Qubits}
\end{center}
\vspace*{0.2cm}

%%%%%%%%%%%%%%%%%%%%%%%%%%%%%%%%%%%%%%%%%%%%%%%%%%%%%%
%% Insert your solutions here %%%%%%%%%%%%%%%%%%%%%%%%
%%%%%%%%%%%%%%%%%%%%%%%%%%%%%%%%%%%%%%%%%%%%%%%%%%%%%%

\Problem{1}
We denote two orthonormal states of a single qubit as $\{\ket{0}, \ket{1}\}$
where
\begin{equation}
\braket{0|0} = \braket{1|1} =1, \quad\quad \braket{0|1}=\braket{1|0}=0.
\end{equation}
Any state $\ket{\Psi}$ of this system can be written as a superposition
\begin{equation}
\ket{\Psi}=\alpha\ket{0} + \beta\ket{1}, \quad |\alpha|^2+|\beta|^2=1, \quad\quad \alpha,\beta \in \mathbb{C}.
\end{equation}

\begin{enumerate}[a)]
    \item Find a parameter representation for $\ket{\Psi}$ if the underlying
    field is (i) the set of real numbers and (ii) the set of complex numbers.

    \item Consider the normalized states
    \begin{equation}
    \ket{\Psi_1}=\vector{\cos\theta_1}{\sin\theta_1}, \quad\quad \ket{\Psi_2}=\vector{\cos\theta_2}{\sin\theta_2}.
    \end{equation}
    Find the condition on $\theta_1$ and $\theta_2$ such that $\ket{\Psi_1}+\ket{\Psi_2}$ is normalized.

    \item Let
    \begin{equation}
    A = \ket{0}\bra{0} + \ket{1}\bra{1}.
    \end{equation}
    Calculate $A$ for
    \begin{enumerate}[(i)]
        \item $\ket{0} = \vector{\cos\theta}{\sin\theta} \quad\quad \ket{1} = \vector{\sin\theta}{-\cos\theta}$
        \item $\ket{0} = \vector{1}{0}, \quad\quad \ket{1} = \vector{0}{1}$
        \item $\ket{0} = \frac{1}{\sqrt 2}\vector{1}{1}, \quad\quad \ket{1} = \frac{1}{\sqrt 2}\vector{1}{-1}$
    \end{enumerate}
\end{enumerate}


\Problem{2}
The {\it Walsh-Hadamard} transform is defined as
\begin{equation}
\ket{0} \rightarrow \frac{1}{\sqrt 2}\left(\ket{0}+\ket{1}\right), \quad\quad \ket{1} \rightarrow \frac{1}{\sqrt 2}\left(\ket{0}-\ket{1}\right)
\end{equation}
\begin{enumerate}[a)]
    \item Find the unitary operator $U_\text{H}$ which implements the {\it Walsh-Hadamard} transform with respect to the basis $\{ \ket{0}, \ket{1} \}$.
    \item Find the inverse of the operator $U_\text{H}$.
    \item Find the matrix representation of $U_\text{H}$ for the basis
    \begin{equation}
    \ket{0} = \vector{\cos\theta}{\sin\theta}, \quad\quad \ket{1} = \vector{\sin\theta}{-\cos\theta}.
    \end{equation}
\end{enumerate}

\Problem{3}
Let $\ket{\Psi} = \vector{e^{i\phi}\cos\theta}{\sin\theta}$ where $\phi,\theta \in \mathbb{R}$.
\begin{enumerate}[a)]
    \item Find $\rho = \ket{\Psi}\bra{\Psi}$.
    \item Find $\Tr{\rho}$.
    \item Find $\Tr \rho^2$.
\end{enumerate}

\Problem{4}
Given the Hamilton operator
\begin{equation}
H = \hbar\omega\sigma_x, \quad\quad \sigma_x = \left(\begin{array}{cc} 0 & 1 \\ 1 & 0 \end{array}\right).
\end{equation}
\begin{enumerate}[a)]
    \item Find the eigenenergies and the eigenstates of the Hamiltonian.
    \item Find the solution $\ket{\Psi(t)}$ for the time-dependent Schrödinger equation
    \begin{equation}
    i\hbar\partial_t \ket{\Psi} = H\ket{\Psi}
    \end{equation}
    with the initial conditions $\ket{\Psi(t=0)} = \vector{1}{0}$.
    \item Find and discuss the probability $\left|\braket{\Psi(t=0)|\Psi(t)}\right|^2$.
\end{enumerate}

\Problem{5}
A system of $n$-qubits represents a finite-dimensional Hilbert space over the
complex numbers of dimension $2^n$. A state $\ket{\Psi}$ is a superposition
of the basic states
\begin{equation}
\ket{\Psi} = \sum_{j_1,j_2,\dots,j_n=0}^1 c_{j_1,j_2,\dots,j_n} \ket{j_1} \otimes \ket{j_2} \otimes \dots \otimes \ket{j_n}
           = \sum_{j_1,j_2,\dots,j_n=0}^1 c_{j_1,j_2,\dots,j_n} \ket{j_1 j_2 \dots j_n}.
\end{equation}
Can the state
\begin{equation}
\ket{\Psi} = \frac{1}{2} \left(\ket{00} + \ket{01} + \ket{10} + \ket{11}\right)
\end{equation}
be written as a product state, i.e. in the form of $\ket{\Psi} = \ket{\Phi_1} \otimes \ket{\Phi_2}$?

%%%%%%%%%%%%%%%%%%%%%%%%%%%%%%%%%%%%%%%%%%%%%%%%%%%%%%
%%%%%%%%%%%%%%%%%%%%%%%%%%%%%%%%%%%%%%%%%%%%%%%%%%%%%%
\end{document}

